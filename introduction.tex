\begin{chapter}{Introduction}
  Reaction Systems (RSs) are a successful computational framework inspired by biological system.
  The interaction between biochemical reactions and the functioning of single reactions are based on the mechanisms of facilitation and inhibition, which can be modeled and analyzed using RS.\

  In this work new software for modeling, analyzing and designing Reaction Systems is designed and developed, with focus on performance and user interface design.

  A Reaction System consists of a set of entities and a set of reactions over them. Each reaction produces some set of entities P (called products) if enabled, meaning if a set R (called reactants) is wholly present and if a set I (called inhibitors) of entities is completely absent.
  The use of inhibitors induces non-monotonic behaviors that are difficult to analyze.
  Entities can also be provided by an external context sequence to simulate \textit{in silico} biological experiment, expanded by structural operational semantics (SOS) rules to account for several biological experiments. In addition Positive RS, trace slicing, graph generation, bisimulation and more is available through an intuitive visual language with a graphical interface.

  

\end{chapter}
